\documentclass[11pt]{article}
\usepackage{geometry}                % See geometry.pdf to learn the layout options. There are lots.
\geometry{letterpaper}                   % ... or a4paper or a5paper or ... 
%\geometry{landscape}                % Activate for for rotated page geometry
\usepackage[parfill]{parskip}    % Activate to begin paragraphs with an empty line rather than an indent
\usepackage{graphicx}
\usepackage{amssymb}
\usepackage{epstopdf}
\usepackage{hyperref}

\DeclareGraphicsRule{.tif}{png}{.png}{`convert #1 `dirname #1`/`basename #1 .tif`.png}

\title{OpenLCB Test plan for the Event Transport Protocol}
\author{The OpenLCB Group}
%\date{}                                         % Activate to display a given date or no date

\begin{document}
\maketitle


\section{Introduction}

This note documents the procedure for testing an OpenLCB implementation against the 
\href{https://nbviewer.org/github/openlcb/documents/blob/master/standards/EventTransportS.pdf}{Event Transport Standard}.

The tests are traceable to specific sections of the Standard.

The testing assumes that the Device Under Test (DUT) is being exercised by other
nodes on the message network, 
e.g. is responding to enquiries from other parts of the message network.

\subsection{Required Equipment}

See the separate ``Installing the OpenLCB Test Software" document for initial installation 
and set up of the test program.

For proper operations, these tests require that only one node be present and communicating.

If a direct CAN connection will be used,
a supported USB-CAN adapter
    \footnote{See ``Installing the OpenLCB Test Software"}
is required. 
Connect the adapter to the DUT using a single UTP cable and connect two CAN terminators.

Provide power to the DUT using its recommended method.

\section{Set Up}
The following steps need to be done once to configure the test program:.
\begin{enumerate}
\item Start the test configuration program. 
\item Select ``Set Up DUT".
\item Get the Node ID from the DUT\footnote{Where do we require this to be marked on a node?} 
\item Enter that Node ID into the program.
\item Configure the test program for the USB-CAN adapter's device address
        or the TCP hostname and port.
\item Quit the test program and reply ``Y" to "Save configuration?" when prompted.
\end{enumerate}

The following steps need to be done at the start of each testing session.
\begin{enumerate}
\item Check that the DUT is ready for operation.
\item Start the test program.
\end{enumerate}

\section{Event Transport Procedure}

Select ``Event testing" in the test program, 
then select each section below in turn.  Follow the prompts
for when to reset/restart the node and when to check 
outputs against the node documentation.

A node which does not self-identify in PIP that it supports
Event Transfer will be deemed to have passed these tests.
\footnote{Using the -p option or setting the checkpip default value False will skip this check.}

A node which does self-identify in PIP that it supports 
Event Transfer is expected to consume or produce at least 
one event.  The tests are structured to check for that.

\textbf{Note:}  Proper handling of known events should be addressed.

\textbf{Note:}  This does not address the proper use of Unique IDs for Event IDs.

\subsection{Identify Events Addressed}

This section tests the addressed interaction in Standard section 6.2 and
the message formats in Standard section 4.3 through 4.8.

The test starts by sending an Identify Events message addressed to the DUT.
It then checks

\begin{enumerate}
\item That one or more Producer Identified, Producer Range Identified, 
        Consumer Identified and/or Consumer Range Identified messages are returned,
\item That those show the DUT node as their source.
\end{enumerate}

\subsection{Identify Events Global}

This section tests the unaddressed (global) interaction in Standard section 6.2 and
the message formats in Standard section 4.3 through 4.8.

The test starts by sending an Identify Events unaddressed (global) message.
It then checks

\begin{enumerate}
\item That one or more Producer Identified, Producer Range Identified, 
        Consumer Identified and/or Consumer Range Identified messages are returned,
\item That those show the DUT node as their source,
\item That these identify the same events produced and consumed as the 
        addressed form of the Identify Events message.
\end{enumerate}

\subsection{Identify Producer}

This section tests the interaction in Standard section 6.3, and
the message formats in Standard section 4.5 through 4.7.

The test proceeds by sending multiple Identify Producers messages for 
the zero or more individual event IDs returned by an Identify Events message 
addressed to the DUT. If there are none of these, this test passes. If there
are one or more, the test then checks:

\begin{enumerate}
\item That exactly one reply is received for each Identify Producers message sent.
\item That those show the DUT node as their source,
\item That these identify the same event ID as the corresponding Identify message.
\end{enumerate}

\subsection{Identify Consumer}

This section tests the interaction in Standard section 6.4, and
the message formats in Standard section 4.2 through 4.4.

The test proceeds by sending multiple Identify Consumers messages for 
the zero or more individual event IDs returned by an Identify Events message 
addressed to the DUT. If there are none of these, this test passes. If there
are one or more, the test then checks:

\begin{enumerate}
\item That exactly one reply is received for each Identify Consumers message sent.
\item That those show the DUT node as their source,
\item That these identify the same event ID as the corresponding Identify message.
\end{enumerate}

\subsection{Initial Advertisement}

Follow the prompts when asked to reset or otherwise initialize the DUT.

This section's tests the interaction in the preamble to Standard section 6, and
the messages in Standard sections 4.1, 4.3, 4.4, 4.6 and 4.7.

\textbf{Note:}  There's no requirement that the Identify Producer messages
be sent immediately, only that they be sent before the events are produced.
Nodes typically send them immediately, and that's what this test checks.

The tests starts by restarting the node, which causes a transition to Initialized
state.  That is then followed by the node identifying events that it will 
produce and consume by appropriate messages. The test then checks:

\begin{enumerate}
\item That the Producer Identified, Producer Identified Range, Consumer Identified 
    and Consumer Identified Range messages produced at node startup are the same
    as the ones emitted in response to an addressed Identify Events,
\item That those messages show the DUT as their source.
\item That no Producer Consumer Event Report messages are sent before the 
    corresponding producer has been identified.
\item That any PCER messages that are sent show the DUT as their source.
\end{enumerate}

\end{document}  
