\documentclass[11pt]{article}
\usepackage{geometry}                % See geometry.pdf to learn the layout options. There are lots.
\geometry{letterpaper}                   % ... or a4paper or a5paper or ... 
%\geometry{landscape}                % Activate for for rotated page geometry
\usepackage[parfill]{parskip}    % Activate to begin paragraphs with an empty line rather than an indent
\usepackage{graphicx}
\usepackage{amssymb}
\usepackage{epstopdf}
\usepackage{hyperref}

\DeclareGraphicsRule{.tif}{png}{.png}{`convert #1 `dirname #1`/`basename #1 .tif`.png}

\title{Installing the OpenLCB Test Software\linebreak{}Basic Version}
\author{The OpenLCB Group}
%\date{}                                         % Activate to display a given date or no date

\begin{document}
\maketitle


\section{Introduction}

This document describes how to obtain and run a set of basic conformance checks for 
OpenLCB nodes.  

The checks are based on the Python `openlch' module.
More information on that can be obtains from its
\href{https://github.com/bobjacobsen/PythonOlcbNode}{GitHub project site.}

For more information on the conformance checks, see the
\href{https://github.com/bobjacobsen/PythonOlcbNode/blob/main/conformance/README.md}{package documentation}
or the 
\href{https://github.com/bobjacobsen/PythonOlcbNode/tree/main/conformance/testplans/}{directory of test plans}.

\section{Obtaining the Software}

The software is distributed as a set of inter-connected Python source files.

\subsection{Obtaining and Using via Git}

If you're using Git, 
\begin{verbatim}
git clone https://github.com/bobjacobsen/PythonOlcbNode.git
\end{verbatim}
will create a PythonOlcbNode directory containing the most recent version of the software.
This also contains git tags for the released versions.

\subsection{Obtaining by Downloading a .zip File}

You can get a download of the most recent released version by going to the project's 
\href{https://github.com/bobjacobsen/PythonOlcbNode/tags}{Github releases web page tag section}
\footnote{Linked above or see \href{https://github.com/bobjacobsen/PythonOlcbNode/tags}{https://github.com/bobjacobsen/PythonOlcbNode/tags}}
and clicking the .zip or .tgz icon on the most recent release.

To get the very most recent version,
\footnote{But if you want to stay current with development of the tools, you should probably be using Git.}
go to the project's
\href{https://github.com/bobjacobsen/PythonOlcbNode}{Github main web page tag section},
click the green Code button, and select "Download Zip".

Expand the downloaded file in a suitable place.


\section{Configuring for Running}

You need to have Python 3.10 installed to run the program. Consult your
computer's documentation for how to install that.  Many computers already
have it installed.

You need to have PYTHONPATH defined to include the main directory.
\footnote{Eventually, this will no longer be necessary, but not quite yet.}
In the Linux and macOS
terminals, you can do this with

\begin{verbatim}
export PYTHONPATH=$PWD
\end{verbatim}
while in the distribution directory (where you installed the code)
or you can place the same line at the end of your startup configuration file.

If you don't add this to your startup configuration file, you'll have to do 
this each time you start a terminal session.

Next
\begin{verbatim}
cd conformance
\end{verbatim}
to get to the right directory for running the code. 

To start the program:
\begin{verbatim}
python3.10 control_master.py
\end{verbatim}

Depending on your Python installation, this simpler form may also work:
\begin{verbatim}
./control_master.py
\end{verbatim}


\section{Configuring the Test Setup}

When you first start the program, you'll be shown a basic menu:

\begin{verbatim}
OpenLCB test program
 0 Setup
 1 Message Network testing
 2 SNIP testing
  
 q  Quit
>> 
\end{verbatim}

Type 0 and hit return to get the setup menu:

\begin{verbatim}
The current settings are:
  hostname = None
  portnumber = 12021
  devicename = None
  targetnodeid = None
  ownnodeid = 03.00.00.00.00.01
  checkpip = True
  trace = 10

c change setting
h help
r return
>> 
\end{verbatim}

At a minimum, you should define how to connect to your OpenLCB network,
and the Node ID of the device you want to test.  

To change the Node ID, select the``change setting" option and work through the prompts:

\begin{verbatim}
>> c
enter variable name
>> targetnodeid
enter new value
>> 02.01.57.00.04.9A
The current settings are:
  hostname = None
  portnumber = 12021
  devicename = None
  targetnodeid = 02.01.57.00.04.9A
  ownnodeid = 03.00.00.00.00.01
  checkpip = True
  trace = 10

c change setting
h help
r return

>> 
\end{verbatim}

Get the proper value from either a label on the device, or from its documentation.
\footnote{Some tests, but not all, can determine the node ID themselves if you leave
    the value as None.}
    
There are currently two ways to connect to your OpenLCB network:
\begin{enumerate}
\item Via a USB-CAN adapter, or
\item Via a GridConnect-format TCP/IP connection.
\end{enumerate}

For a USB-CAN connector, define the devicename to be the address of the device in your computer, 
e.g. /dev/cu.usbmodemCC570001B1 or COM7.

For a TCP/IP link, define the hostname to be the IP address or host name to be used 
for connecting.

You must specify one or the other of hostname and device name, but not both.
When you enter one, the other will be set to None.

When done with setup, select return.  You'll be asked if you want to save changes.  
Select y to save and n to skip saving.

\begin{verbatim}
>> r

Do you want to save the new settings? (y/n)
 >> y
Stored
Quit and restart the program to put them into effect

OpenLCB test program
 0 Setup
 1 Message Network testing
 2 SNIP testing
  
 q  Quit
>> 
\end{verbatim}


\end{document}  
