\documentclass[11pt]{article}
\usepackage{geometry}                % See geometry.pdf to learn the layout options. There are lots.
\geometry{letterpaper}                   % ... or a4paper or a5paper or ... 
%\geometry{landscape}                % Activate for for rotated page geometry
\usepackage[parfill]{parskip}    % Activate to begin paragraphs with an empty line rather than an indent
\usepackage{graphicx}
\usepackage{amssymb}
\usepackage{epstopdf}
\usepackage{hyperref}

\DeclareGraphicsRule{.tif}{png}{.png}{`convert #1 `dirname #1`/`basename #1 .tif`.png}

\title{Checking the OpenLCB Configuration Description Information Standard}
\author{The OpenLCB Group}
%\date{}                                         % Activate to display a given date or no date

\begin{document}
\maketitle


\section{Introduction}

This note documents the procedure for checking an OpenLCB implementation against the 
\href{https://nbviewer.org/github/openlcb/documents/blob/master/standards/ConfigurationDescriptionInformationS.pdf}
    {Configuration Description Information Standard}.

The checks are traceable to specific sections of the Standard.

The checking assumes that the Device Being Checked (DBC) is being exercised by other
nodes on the message network, 
e.g. is responding to enquiries from other parts of the message network.



\section{Configuration Description Information Procedure}

Select ``CDI checking" in the program, 
then select each section below in turn.

A node which does not self-identify in PIP that it supports
Configuration Description Information will be deemed to have passed these checks.
\footnote{Using the -p option or setting the checkpip default value False will skip this check.}

This plan assumes that the Datagram Transport Protocol and the Memory Configuration Protocol 
have been separately checked. It uses those, but does not do any detailed checking of them.

\subsection{Validation checking}

This section checks the content of the CDI to make sure that it is valid XML. 
It reads the information from the 0xFF memory space, 
then validates it against the 1.3 XML Schema which is stored in a local ”schema.xsd” file.

\end{document}  
