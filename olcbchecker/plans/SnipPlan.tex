\documentclass[11pt]{article}
\usepackage{geometry}                % See geometry.pdf to learn the layout options. There are lots.
\geometry{letterpaper}                   % ... or a4paper or a5paper or ... 
%\geometry{landscape}                % Activate for for rotated page geometry
\usepackage[parfill]{parskip}    % Activate to begin paragraphs with an empty line rather than an indent
\usepackage{graphicx}
\usepackage{amssymb}
\usepackage{epstopdf}
\usepackage{hyperref}

\DeclareGraphicsRule{.tif}{png}{.png}{`convert #1 `dirname #1`/`basename #1 .tif`.png}

\title{Checking the OpenLCB Simple Node Information Protocol Standard}
\author{The OpenLCB Group}
%\date{}                                         % Activate to display a given date or no date

\begin{document}
\maketitle


\section{Introduction}

This note documents the procedure for checking an OpenLCB implementation against the 
\href{https://nbviewer.org/github/openlcb/documents/blob/master/standards/SimpleNodeInformationS.pdf}
    {Simple Node Information Protocol Standard}.

The checks are traceable to specific sections of the Standard.

The checking assumes that the Device Being Checked (DBC) is being exercised by other
nodes on the message network, 
e.g. is responding to enquiries from other parts of the message network.



\section{Simple Node Information Protocol Procedure}

Select ``SNIP Checking" in the program, 
then select each section below in turn.  Follow the prompts
for when to reset/restart the node and when to check 
outputs against the node documentation.

A node which does not self-identify in PIP that it supports
the Simple Node Information Protocol will be deemed to have passed these checks.
\footnote{Using the -p option or setting the checkpip default value False will skip this check.}

\subsection{SNIP reply checking}

This section checks the format of the reply message in Sections 4.2 and 5.1 of the Standard.

It does this by issuing a Simple Node Information Request message, 
accumulating the reply(s), and then checking:
    \begin{enumerate}
    \item The message indicates its source is the DBC.
    \item The message indicates its destination is the checking node.
    \item The version byte at the start of the first section is either 1 or 4.
    \item The version byte at the start of the second section is either 1 or 2.
    \item There are exactly six zero bytes.
    \item Each of the six defined strings is no longer than its defined maximum length.
    \item There are no data byte(s) after the sixth zero byte.
    \end{enumerate}

\end{document}  
