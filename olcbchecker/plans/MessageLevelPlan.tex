\documentclass[11pt]{article}
\usepackage{geometry}                % See geometry.pdf to learn the layout options. There are lots.
\geometry{letterpaper}                   % ... or a4paper or a5paper or ... 
%\geometry{landscape}                % Activate for for rotated page geometry
\usepackage[parfill]{parskip}    % Activate to begin paragraphs with an empty line rather than an indent
\usepackage{graphicx}
\usepackage{amssymb}
\usepackage{epstopdf}
\usepackage{hyperref}

\DeclareGraphicsRule{.tif}{png}{.png}{`convert #1 `dirname #1`/`basename #1 .tif`.png}

\title{Checking the OpenLCB Message Network Standard}
\author{The OpenLCB Group}
%\date{}                                         % Activate to display a given date or no date

\begin{document}
\maketitle


\section{Introduction}

This note documents the procedure for checking an OpenLCB implementation against the 
\href{https://nbviewer.org/github/openlcb/documents/blob/master/standards/MessageNetworkS.pdf}
    {OpenLCB Message Network Standard}.

The checks are traceable to specific sections of the Standard.

The checking assumes that the Device Being Checked (DBC) is being exercised by other
nodes on the message network, 
e.g. is responding to enquiries from other parts of the message network.

\subsection{Required Equipment}

See the separate ``Installing the OpenLCB Checker Software" document for initial installation 
and set up of the checker program.

If a direct CAN connection will be used,
a supported USB-CAN adapter
    \footnote{See ``Installing the OpenLCB Checker Software"}
is required. 
Connect the adapter to the DBC using a single UTP cable and connect two CAN terminators.

Provide power to the DBC using its recommended method.

\section{Set Up}
The following steps need to be done once to configure the checker program:.
\begin{enumerate}
\item Start the checker configuration program. 
\item Select ``Set Up".
\item Get the Node ID from the DBC\footnote{Where do we require this to be marked on a node?} 
\item Enter that Node ID into the program.
\item Configure the checker program for the USB-CAN adapter's device address
        or the TCP hostname and port.
\item Quit the checker program and reply ``Y" to "Save configuration?" when prompted.
\end{enumerate}

The following steps need to be done at the start of each checking session.
\begin{enumerate}
\item Check that the DBC is ready for operation.
\item Start the checker program.
\end{enumerate}

\section{Message Network Procedure}

Select ``Message Network checking" in the program, 
then select each section below in turn.  Follow the prompts
for when to reset/restart the node and when to check 
outputs against the node documentation.

\subsection{Node Initialized checking}

This section checks the interaction in Standard section 3.4.1 Node Initialization.

It does this by having the operator reset/restart the DBC and then
checking that a Node Initialized message is received and
    \begin{enumerate}
    \item The Node Initialized message is the first message received.
            \footnote{See Section 3.2 of the Standard}
    \item The message indicates its source is the DBC.
    \item The message uses the appropriate MTI
            \footnote{See Section 3.3.1 of the Standard}
            for whether PIP indicates the node 
            uses the Simple Protocol or not.
    \item The data part of the message contains the source node ID.
    \end{enumerate}


\subsection{Verify Node checking}

This section checks the interaction in Standard section 3.4.2 Node ID Detection.

It does this by

\begin{enumerate}
\item Sending a global Verify Node ID message and checking the reply.
    \begin{enumerate}
    \item The reply message indicates its source is the DBC.
    \item The reply message contains the Node ID of the DBC in its data field.
    \item The reply message uses the appropriate MTI
            \footnote{See Section 3.3.3 fo the Standard}
            for whether PIP indicates the node 
            uses the Simple Protocol or not.
    \end{enumerate}

\item Sending a Verify Node ID message addressed to the DBC and checking the reply.
    \begin{enumerate}
    \item The reply message indicates its source is the DBC.
    \item The reply message contains the Node ID of the DBC in its data field.
    \item The reply messages uses the appropriate MTI for whether PIP indicates the node 
            uses the Simple Protocol or not.
    \end{enumerate}

\item Sending a Verify Node ID message addressed to a different address from the DBC
        and checking for the absence of a reply.
\end{enumerate}

\subsection{Protocol Support Inquiry checking}

This section checks the interaction in Standard section 3.4.3 Protocol Support Inquiry and Response.

It does this by 
\begin{enumerate}
\item Sending a Protocol Support Inquiry message to the node, 
and checking for the resulting Protocol Support Response message.  
    \begin{enumerate}
    \item The reply message is addressed to the checking node.
    \item The reply message indicates its source is the DBC.
    \end{enumerate}
The check displays the resulting list of supported protocols for checking against the 
DBC's documentation.

\item Sending a Protocol Support Inquiry message addressed to a different address from the DBC
        and checking for the absence of a reply.
\end{enumerate}


\subsection{Optional Interaction Rejected checking}

This section checks the interaction in Standard section 3.5.1 Reject Addressed Optional Interaction.

It does this by sending an unallocated MTI
\footnote{Initially 0x048. Others may be added in the future for an eventual
    check of all possible unallocated MTIs}
addressed to the DBC and checking for the Optional Interaction Rejected 
message in reply.

    \begin{enumerate}
    \item The reply message is addressed to the checking node.
    \item The reply message indicates its source is the DBC.
    \item The reply message carries at least four bytes of content, with the 
            third and fourth bytes carrying the original MTI.
    \end{enumerate}

\subsection{Duplicate Node ID Discovery checking}

This section checks the interaction in Standard section 3.5.4 Duplicate Node ID Discovery.

It does this by sending a global Verify Node message with the source ID set to the DBC's ID.

It then checks for the well-known global event.  If it finds that, the check passes.
If not, it prompts the operator to confirm that indication has been made 
"using whatever indication technology is available", e.g. via LEDs.

\section{Frame Level Procedure}

This section is only applicable to implementations what use a CAN-format link layer
implementation, e.g. via a USB-CAN adapter or by using Grid Connect coding
over a TCP/IP link. See Standard section 7.

All checks in the prior section are assumed to have passed before these
checks are run, as they depend on message-layer behaviors.

In general, proper operation of the frame level is checked by 
exercising it through the message level in the checks above and in subsequent 
documents.  We limit ourselves here to checking atypical behaviors.

\subsection{Reserved Field Handling}

This section discusses the reserved bit fields described in Standard section 3.1.1.

The CAN implementation does not carry bits 15-14 nor bit 12 of the MTI. 
The full 12 bit sof the rest of the MTI are checked in other sections of this check plan.

However, in a CAN implementation, the 0x1\_0000\_0000 bit is reserved, send as 1, 
don't check on receipt.

To check this, the checker sends a Verify Node ID Global with zero in the 0x1\_0000\_0000
bit. It then checks:

\begin{enumerate}
\item That a Verified Node ID message frame is received from the DBC,
\item That the 1\_0000\_0000 bit in that frame is a one.
\end{enumerate}



\end{document}  
