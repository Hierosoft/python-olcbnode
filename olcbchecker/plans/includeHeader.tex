The checks are traceable to specific sections of the Standard.

The checking assumes that the Device Being Checked (DBC) is being exercised by other
nodes on the message network, 
e.g. is responding to enquiries from other parts of the message network.

\subsection{Required Equipment}

See the separate ``Installing the OpenLCB Checker Software" document for initial installation 
and set up of the checker program.

If a direct CAN connection will be used,
a supported USB-CAN adapter
    \footnote{See ``Installing the OpenLCB Checker Software"}
is required. 
Connect the adapter to the DBC using a single UTP cable and connect two CAN terminators.

Provide power to the DBC using its recommended method.

\section{Set Up}
The following steps need to be done once to configure the checker program:.
\begin{enumerate}
\item Start the checker configuration program. 
\item Select ``Set Up".
\item Get the Node ID from the DBC\footnote{Where do we require this to be marked on a node?} 
\item Enter that Node ID into the program.
\item Configure the checker program for the USB-CAN adapter's device address
        or the TCP hostname and port.
\item Return from the setup section and reply ``Y" to "Save configuration?" when prompted.
\item Quit from the program.
\end{enumerate}

The following steps need to be done at the start of each checking session.
\begin{enumerate}
\item Check that the DBC is ready for operation.
\item Start the checker program.
\end{enumerate}
